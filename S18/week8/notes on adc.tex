\documentclass[letter,10pt]{article}

\usepackage{amsmath}
%\usepackage[utf8]{inputenc}
\usepackage[english]{babel}
%\usepackage[ngerman,english]{babel}
%\usepackage{hyperref}
%\usepackage{graphicx}
%\usepackage{listings}
%\usepackage{adjustbox}
%\usepackage[affil-it]{authblk}
%\usepackage[T1]{fontenc}
%\usepackage{lmodern}
%\usepackage[space]{grffile}
\usepackage{hyperref}

\begin{document}



\section*{Note on Analog-to-Digital Converter (ADC)}

``What does the ADC output mean? What are those big numbers?''

\subsection*{Resolution}
An ADC linearly maps an input voltage to a range of numbers. The ADC we use has a 5V reference and is a 10-bit ADC, so it maps an input voltage between 0 and 5V to a number between 0 to $2^{10} - 1$:
\begin{equation}
ADC = V_{input} / V_{ref} * 2^{10} = V_{input} / 5V * 1024
\end{equation}
(Reality is a bit more complicated than that. Divide by 1023 or 1024, which is correct? Or does it matter?)\\
\\
For example, if the ADC outputs 365, then the voltage of the input is
\begin{equation}
\frac{365}{1024}*5V = 1.78V
\end{equation}
You should verify this with a voltmeter. Also try to verify that the reference voltage is indeed (not exactly) 5V. How does that affect your measurement?

\end{document}
